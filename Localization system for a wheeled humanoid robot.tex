% % % Document class
\documentclass{report}
%\documentclass{article}
% "article" = page number on title page
% "report" = no page number on title page


% % % Naming conventions
% % Cross references
% Equations - \label{e:label} \eref{e:label}
% Sections - \label{s:label} \ref{s:label}
% Subsections - \label{s:mainlabel--subsectionlabel} \ref{s:mainlabel--subsectionlabel}
% Figures - \label{f:label} \fref{label}
% % Numbering
% Sections - n * 10; n = 1, 2, ... - gives the possibility to shift other sections in between
% Equations - 
% Figures - 

% % % User packages
% % Languages
\usepackage{amsmath,amsfonts,amsthm,amssymb}
\usepackage[T1]{fontenc}
\usepackage{polski}
\usepackage[utf8]{inputenc}
\usepackage[polish,german,english]{babel}
% % General
\usepackage[usenames,dvipsnames]{color}
\usepackage{extramarks}
\usepackage{fancyhdr}
\usepackage{float}
\usepackage{graphicx}
\usepackage{hyperref}
\usepackage{chngpage}
\usepackage{soul}
\usepackage[retainorgcmds]{IEEEtrantools}
\usepackage{ifthen}
\usepackage{lastpage}
\usepackage{listings}
\usepackage{cancel}
\usepackage{courier}
\usepackage{layout}
\usepackage{lscape}
\usepackage{media9}
\usepackage{multicol}
\usepackage{multimedia}
\usepackage{multirow}
\usepackage{pdflscape}
\usepackage{setspace}
\usepackage{tabularx}
\usepackage{varioref}
\usepackage{wrapfig}
\usepackage{xcolor}
\usepackage{xfrac}
\usepackage{xstring}


% % % Packages configuration
% % Tabular
\newcolumntype{Y}{>{\centering\arraybackslash}X}


% % % Command definitions
% % References
\newcommand{\bref}[1]{(\ref{#1})}
\newcommand{\cref}[1]{(\afig\ref{#1})}
%\renewcommand{\eqref}[1]{(\ref{eq:#1})}
\newcommand{\eref}[1]{(\ref{#1})}
\newcommand{\fref}[1]{(\ref{f:#1})}
\newcommand{\sref}[1]{\ref{#1}}
% % Symbols
\newcommand{\dlsh}{\reflectbox{\rotatebox[origin=c]{180}{$\Lsh$}}}
\newcommand{\drsh}{\reflectbox{\rotatebox[origin=c]{180}{$\Rsh$}}}
\newcommand\ccancel[2][black]{\renewcommand\CancelColor{\color{#1}}\cancel{#2}}
\newcommand\ccancelto[3][black]{\renewcommand\CancelColor{\color{#1}}\cancelto{#2}{#3}}
% % Font styles
\newcommand\dunderline[1]{\underline{\underline{\ #1\ }}}
\newcommand\sunderline[1]{\underline{\ #1\ }}
% % Mathematical symbols and structures
%\renewcommand{\vec}[1]{\overrightarrow{\mathbf{#1}}}
\renewcommand{\vec}[1]{\mathbf{\overline{#1}}}
% % Other
\StrLeft{\figurename}{1}[\afig]

% % Complex definitions
% Links
% Generic hyperlink -- \link[optional text]{link}
\newcommand*{\link}[2][]{
	\ifthenelse{\equal{#1}{}}
	{\href{#2}{#2}}
	{\href{#2}{#1}}}

% Http hyperlink -- \httplink[optional text]{link}
\newcommand*{\httplink}[2][]{
	\ifthenelse{\equal{#1}{}}
	{\href{http://#2}{#2}}
	{\href{http://#2}{#1}}}

% Email hyperlink -- \emaillink[optional text]{link}
\newcommand*{\emaillink}[2][]{
	\ifthenelse{\equal{#1}{}}
	{\href{mailto:#2}{#2}}
	{\href{mailto:#2}{#1}}}

% Graphics
% Inkscape path
%\IfFileExists{/dev/null}{
%	\newcommand{\Inkscape}{/usr/bin/inkscape }}{
%	\newcommand{\Inkscape}{"C:/Program Files (x86)/Inkscape/inkscape.exe" }
%}

% Include SVG withing scalebox
%\includesvga[0.75]{Graphics/10-odometry--right-turn}
%\includesvga[scalebox]{input-files-w/o-ext}
\newcommand{\includesvgsb}[2][1]{
%	\IfFileExists{\Inkscape}{
%		\immediate\write18{\Inkscape -z -D --file="#2.svg" --export-pdf="#2.pdf" --export-dpi=72 --export-latex}}{}
	\scalebox{#1}{\input{#2.pdf_tex}}
}

% Include SVG as figure
%\includesvg[0.75]{a}{Automated includesvg pdf\label{f:f4b}}
%\includesvg[scale]{input-files-w/o-ext}{caption\label{f:id}}
\newcommand{\includesvg}[3][1]{
%	\IfFileExists{\Inkscape}{
%		\immediate\write18{/home/emeres/.bin/testbell}
%		\immediate\write18{\Inkscape -z -D --file="#2.svg" --export-pdf="#2.pdf" --export-dpi=96 --export-latex}}{}
	\begin{figure}[!ht]
%		\centering
		\begin{center}
			\def\svgwidth{#1\columnwidth}
			\input{#2.pdf_tex}
			\caption{#3}
		\end{center}
	\end{figure}
}

% Scaled figure
\newcommand{\scalefig}[3]{
	\begin{figure}[!ht]
%		\centering
		\begin{center}
			\includegraphics[width=#2\columnwidth]{#1}
			\caption{#3}
		\end{center}
	\end{figure}
}

% Scaled and trimmed figure
\newcommand{\scalefigcrop}[7]{
	\begin{figure}[!ht]
%		\centering
		\begin{center}
			\includegraphics[trim=#4 #5 #6 #7, clip, width=#2\columnwidth]{#1}
			\caption{#3}
		\end{center}
	\end{figure}
}

% Script listing
\newcommand{\script}[2]
{\lstinputlisting[caption=#2,label=#1]{#1.m}}
%{\begin{itemize}\item[]\lstinputlisting[caption=#2,label=#1]{#1.m}\end{itemize}}
%{\begin{itemize}\item[]\lstinputlisting[caption=#2,label=#1]{#1.sc}\end{itemize}}
%{\label{sc:#1}\begin{itemize}\item[]\lstinputlisting[caption=#2,label=#1]{#1.sc}\end{itemize}}
%\newcommand{\script}[2]
%{\begin{itemize}\item[]\lstinputlisting[caption=#2,label={sc:#1}]{#1}\end{itemize}}

\lstloadlanguages{Matlab}
\lstset{language=Matlab,
	frame=single,	% Frame type around code
	basicstyle=\small\ttfamily,	% Basic code style
	keywordstyle=[1]\color{Blue}\bf,	% Python functions bold and blue
	keywordstyle=[2]\color{Purple},	% Python function arguments purple
	keywordstyle=[3]\color{Blue}\underbar,	% User functions underlined and blue
	identifierstyle=,	% Nothing special about identifiers
	commentstyle=\usefont{T1}{pcr}{m}{sl}\color{DarkGreen}\small,	% Comment style
	stringstyle=\color{Purple},	% String style
	showstringspaces=false,	% Space string indication
	tabsize=2,
%	escapechar={},
	breaklines=true,
	breakautoindent=false,
	prebreak=\raisebox{0ex}[0ex][0ex]{\ensuremath{\space\color{red}{\blacktriangledown}\color{blue}\thelstnumber}},
	postbreak=\raisebox{0ex}[0ex][0ex]{\ensuremath{\color{blue}\thelstnumber\color{red}{\blacktriangle}\space}},
	morekeywords={xlim, ylim, },	% Not default functions
	morekeywords=[2]{on, off},	% Custom function parameters
	morekeywords=[3]{FindESS},	% Custom functions
	morecomment=[l][\color{Blue}]{...},	% Line continuation (...) like blue comment
	numbers=left,
	numberfirstline=true,
	firstnumber=1,	% Line number offset
	numberstyle=\tiny\color{Blue},	% Line numbers
	stepnumber=5	% Line numbers step
}


% % % Configuration
\graphicspath{{./}{Graphics/}}
%\renewcommand{\figurename}{Wykres}
%\renewcommand{\figurename}{Figure}
%\renewcommand{\figurename}{Image}
%\renewcommand{\tablename}{Tabela}
%\renewcommand{\lstlistingname}{Skrypt}
\renewcommand{\lstlistingname}{Script}


% % % Specific Information
\newcommand{\Code}{86805}
\newcommand{\Title}{Software Architectures for Robotics}
\newcommand{\SubTitle}{Localization system for a wheeled humanoid robot}
\newcommand{\Date}{01.02.16}
\newcommand{\Time}{Wed 23:55}
\newcommand{\Recipient}{Prof. F. Mastrogiovanni, PhD C. Recchiuto}
\newcommand{\Author}{Rabbia Asghar, BEng, Ernest Skrzypczyk, BSc}


% % % Document margins
\topmargin = -0.45in
\evensidemargin = 0in
\oddsidemargin = 0in
\textwidth = 6.5in
\textheight = 9.0in
\headsep = 0.25in


% % % Setup the header and footer
\pagestyle{fancy}
\lhead{\Author}
\chead{\Code: \Title, \Recipient\\} %(\Time,\ \Date)}  %
\rhead{\firstxmark}
\lfoot{\lastxmark}
\cfoot{}
\rfoot{Page\ \thepage/\protect\pageref{LastPage}}
\renewcommand\headrulewidth{0.4pt}
\renewcommand\footrulewidth{0.4pt}
\setlength{\parindent}{4ex}


% % % Colors
\definecolor{DarkGreen}{rgb}{0.0,0.4,0.0}


% % % Debugging
%\tracingall


% % % % CLEAN UP
% % % Style specific commands and configuration
%\newcommand{\enterProblemHeader}[1]{\nobreak\extramarks{#1}{#1 kontynuowany na następnej stroniecontinued on next page\ldots}\nobreak%
%												\nobreak\extramarks{#1 (continued)}{#1 continued on next page\ldots}\nobreak}
%\newcommand{\exitProblemHeader}[1]{\nobreak\extramarks{#1 (continued)}{#1 continued on next page\ldots}\nobreak%
\newcommand{\enterProblemHeader}[1]{\nobreak\extramarks{#1}{#1\ldots}\nobreak%
												\nobreak\extramarks{#1 (continued)}{#1}\nobreak}
\newcommand{\exitProblemHeader}[1]{\nobreak\extramarks{#1 (continued)}{#1}\nobreak%
											  \nobreak\extramarks{#1}{}\nobreak}

\newlength{\labelLength}
\newcommand{\labelAnswer}[2]
  {\settowidth{\labelLength}{#1}
	\addtolength{\labelLength}{0.25in}
	\changetext{}{-\labelLength}{}{}{}
	\noindent\fbox{\begin{minipage}[c]{\columnwidth}#2\end{minipage}}
	\marginpar{\fbox{#1}}

	% We put the blank space above in order to make sure this
	% \marginpar gets correctly placed.
	\changetext{}{+\labelLength}{}{}{}}

\setcounter{secnumdepth}{0}
\newcommand{\ProblemName}{}
\newcounter{ProblemCounter}
\newenvironment{Problem}[1][Problem \arabic{ProblemCounter}]%
	{\stepcounter{ProblemCounter}
	\renewcommand{\ProblemName}{#1}
	\section{\ProblemName}
	\enterProblemHeader{\ProblemName}}
	{\exitProblemHeader{\ProblemName}}

\newcommand{\problemAnswer}[1]
	{\noindent\fbox{\begin{minipage}[c]{\columnwidth}#1\end{minipage}}}

\newcommand{\problemLAnswer}[1]
	{\labelAnswer{\ProblemName}{#1}}

\newcommand{\SectionName}{}
\newlength{\SectionLabelLength}{}
\newenvironment{Section}[1]%
	{ %
%
	\renewcommand{\SectionName}{#1}
	\settowidth{\SectionLabelLength}{\SectionName}
	\addtolength{\SectionLabelLength}{0.25in}
	\changetext{}{-\SectionLabelLength}{}{}{}
	\subsection{\SectionName}
	\enterProblemHeader{\ProblemName\ [\SectionName]}}
	{\enterProblemHeader{\ProblemName}

	\changetext{}{+\SectionLabelLength}{}{}{}}

\newcommand{\sectionAnswer}[1]
{

	\noindent\fbox{\begin{minipage}[c]{\columnwidth}#1\end{minipage}}
	\enterProblemHeader{\ProblemName}\exitProblemHeader{\ProblemName}
	\marginpar{\fbox{\SectionName}}

}

%% Edits the caption width
%\newcommand{\captionwidth}[1]{%
%  \dimen0=\columnwidth	\advance\dimen0 by-#1\relax
%  \divide\dimen0 by2
%  \advance\leftskip by\dimen0
%  \advance\rightskip by\dimen0
%}
% % % % CLEAN UP


% % % % CLEAN UP
% % % Title page
\title{\vspace{2in}\textmd{\textbf{\Code:\ 
\Title\ifthenelse{\equal{\SubTitle}{}}{}{\\\SubTitle}}}\\\normalsize\vspace{0.1in}\small{Date:\ 
\Time,\ \Date}\\\vspace{0.1in}\large{\textit{\Recipient}}\vspace{3in}}
\date{}
\author{\textbf{\Author}}
% % % % CLEAN UP


% % % Document
\begin{document}
\begin{spacing}{1.1}
\maketitle


% % % Table of contents
\setcounter{tocdepth}{1}
\tableofcontents
\newpage
\clearpage

% % % Text
\begin{Problem}[Rollo - Humanoid robot]
\section{Odometry}\label{s:10-odometry}
\indent\indent
Odometry is currently the most widely used technique for determining the position of a mobile robot. %TODO This would requiere a source
It mainly involves use of various encoders, for example on wheels, as sensors to estimate the robot's position relative to a starting or previous location. Usually, it is used for real-time positioning in the between the periodic absolute position measurements, for example \link[GPS (Global Positioning System)]{http://www.gps.gov/} provides absolute position feedback, however it updates at $0.1 \div 1 s$ interval, during which odometry could be used for localization. %ALT / for estimating current position
% Odometry is used to estimate position during its update interval. %REP
%The reason for its popularity is it's simplicity and the straight forward approach ot its implementation. However, this comes at a cost
One of the major downsides of odometry is its sensitivity to errors. There are various error sources discussed in section \nameref{s:10-odometry--errors} \vpageref{s:10-odometry--errors} in detail. One significant source of error influencing the accuracy of odometry that is worth mentioning however, is the integration of velocity measurements over time to give position estimates.

First the odometry based motion model for the robot will be derived.

The model is derived based on the following important assumptions:

\begin{enumerate}
	\item The robot is a rigid body
	\item The model represents a differential drive robot
	\item There is no slip in the wheels
	\item Both wheels are turning in the forward direction
\end{enumerate} 

A differential drive robot runs straight when the linear speed of both the left and right wheel is same. If the speed of one wheel is greater than the other, the robot runs in an arc. This derivation can be divided in three distinct cases of robot motion:

The basic premise for the odometry model of the Rollo humanoid robot is presented in figures \fref{10-odometry-rt-2w} and \fref{10-odometry-lt-4w}.

%\includesvg[0.75]{Graphics/10-odometry--left-turn}{Odometry model with the simplified 2 wheel configuration for the right turn.\label{f:10-odometry-lt-2w}}
\includesvg[0.75]{Graphics/10-odometry--right-turn}{Odometry model with the simplified 2 wheel configuration for the right turn.\label{f:10-odometry-rt-2w}}

%\includesvg[0.75]{Graphics/10-odometry--right-turn--4-wheels}{Odometry model with the simplified 4 wheel configuration for the right turn.\label{f:10-odometry-rt-4w}}
\includesvg[0.75]{Graphics/10-odometry--left-turn--4-wheels}{Odometry model with the simplified 4 wheel configuration for the left turn.\label{f:10-odometry-lt-4w}}

\begin{enumerate}
	\item Clockwise direction
	\item Counterclockwise direction
	\item Straight line
\end{enumerate}

\subsection{Clockwise direction}\label{s:10-odometry--cw}

First, the case is considered when the speed of left wheel is greater than the right, and the robot will run in clockwise direction. Both right and left wheel will rotate around the same center of a circle. %TODO How about making the origin of the circles in the drawings and refer to them?

$P_i$ represents the initial position of the robot, defined by the center of the line joining two wheels, while $l$ represents axle length, ergo distance between two wheels. $S_L$ and $S_R$ represent the distance travelled by left and right wheel respectively.
$\Theta$ represents the angle of travel for both the wheels and $r$ is the radius of travel from the center of robot. 
With encoder feedback from left and right wheel, $S_L$ and $S_R$ can simply be computed using equations \eref{e:SL} and \eref{e:SR}

\begin{equation}\label{e:SL}
	S_L = \frac{n_L}{60} 2 \pi r_L
\end{equation}

\begin{equation}\label{e:SR}
	S_R = \frac{n_R}{60} 2 \pi r_R
\end{equation}

where $n_L$ and $n_R$ are the revolutions per minute ([rpm]) of left and right wheel, and $r_L$ and $r_R$ are the radii of the 2 wheels. %TODO legend instead of essay on meaning?

$S_L$ and $S_R$ can be related to $r$ and $\Theta$ using formulas \eref{e:SL=rtheta} and \eref{e:SR=rtheta}

\begin{equation}\label{e:SL=rtheta}
	S_L = (r + \frac{l}{2}) \Theta
\end{equation}

\begin{equation}\label{e:SR=rtheta}
	S_R = (r - \frac{l}{2}) \Theta 
\end{equation}

The equations \eref{e:SL=rtheta} and \eref{e:SR=rtheta} \vpageref[above]{e:SL=rtheta} can be solved simultaneously to compute $r$ and $l$.

\begin{equation}\label{e:r}
	r  = \frac{l}{2} \cdot \frac{S_L + S_R}{S_L - S_R} % Travel distance
\end{equation}

\begin{equation}\label{e:theta}
	\Theta = \frac{S_R}{r - \frac{l}{2}} % Travel angle
\end{equation}

Final position of the robot and its orientation can then be derived using basic trigonometry and geometry relations as displayed in equation \eref{e:P_fCW}:

\begin{IEEEeqnarray}{CCC}\label{e:P_fCW}
	P_f  & = & 
	\begin{bmatrix}
		P_{ix} + r ( 1 - \cos(\Theta) ), & P_{iy} + r \sin(\Theta)
	\end{bmatrix}
\end{IEEEeqnarray}

\begin{equation*}\label{e:finalorientationCW}
	\mathrm{final\ orientation = initial\ orientation - \Theta}\nonumber
\end{equation*}

\subsection{Counterclockwise direction}\label{s:10-odometry--ccw}
Similar derivation of the robot can be derived for counterclockwise rotation, when the rpm of right wheel is higher than that of the left wheel. The relations for final position and rotation are as follows for this case:

\begin{IEEEeqnarray}{CCC}\label{e:P_fCCW}
	P_f  & = & 
	\begin{bmatrix}
		P_{ix} - r( 1 - \cos(\Theta) ), & P_{iy} + r \sin(\Theta)
	\end{bmatrix}
\end{IEEEeqnarray}

\begin{equation*}\label{e:finalorientationCCW}
	\mathrm{final\ orientation = initial\ orientation + \Theta}
\end{equation*}

\subsection{Straight line motion}\label{s:10-odometry--straight-line-motion}

\begin{IEEEeqnarray}{CCC} \label{e:P_fSt}
	P_f  & = & 
	\begin{bmatrix}
		P_{ix} + S_L( \cos(\Theta) ), & P_{iy} + S_L \sin(\Theta)
	\end{bmatrix}
\end{IEEEeqnarray}

\begin{equation*}\label{e:finalorientationSt}
	\mathrm{final\ orientation = initial\ orientation}
\end{equation*}

\subsection{Adaption of odometry model for rollo}\label{s:10-odometry--rollo}
\indent\indent

Currently, the encoders feedback is unavailable in the robot and true odometry model can not be implemented.
However, an attempt has been made to implement its modified version.
The distance covered by right and left wheels is instead estimated from the control command.

\subsubsection{Straight line motion}\label{s:10-odometry--rollo--straight-line-motion}
Because of the mechanical difference in the two legs, it is almost impossible for the two wheels to run at same speed . Moreover, the two motors/wheels are powered using 2 different LIPO packs and are run in open loop. The difference in voltage level further results in different speed of the motor at the same pwm. This is a major source for deviation in behavior of robot's adjusted odometry model.

Rollo was run in a straight line at different commands and its location feedback from the motion capture system was logged. For a certain command, average speed of left and right wheel was determined from the logs.


  
\subsection{Odometry errors}\label{s:10-odometry--errors} %TODO refine the whole subsection, define systematic and random errors, shows the differences and the influence on the overall error
\indent\indent
This method is sensitive to errors  Rapid and accurate data collection, equipment calibration, and processing are required in most cases for odometry to be used effectively  straight forward to implement

Odometry Error Sources:
\begin{enumerate}
	\item Limited resolution during integration (time increments, measurement resolution).
	\item Unequal wheel diameter (deterministic)
	\item Variation in the contact point of the wheel (deterministic)
	\item Unequal floor contact and variable friction can lead to slipping (non deterministic)
\end{enumerate}




The errors can be divided into systematic and random. 
Three main sources of systematic errors in odometry:
\begin{itemize}
	\item Distance
	\item Rotation
	\item Skew
\end{itemize}

Last two more significant with time.



\section{System and Measurements Model}\label{s:20-Model}
\indent\indent
\subsection{System Model}\label{s:20-Model--SysModel}
In order to implement Kalman Filter, the previously described model must be first represented in state space representation.

We can define the location of the robot at instant $k$ using state variables. This will include position in x and y coordinates and orientation.
\begin{IEEEeqnarray}{CCCCCCCCC} \label{eq:xk localistion}
	x_k & = & 
	\begin{bmatrix}
		x_{[x],k}\\
		x_{[y],k}\\
		x_{[\phi],k}\\
	\end{bmatrix}
\end{IEEEeqnarray}

The control input provided to robot is the speed of right and left wheel. This defines the distance covered by both the wheels in unit time. The relative displacement of the robot at instant $k$ can be notated by $d_k$. Using equations \ref{e:P_fCCW},\ref{e:P_fCW} and \ref{e:P_fSt}, relative displacement can be expressed in terms of $r$ and $\theta$. Thus, control input $u_k$ can be expressed as a function of relative displacement.

\begin{equation} \label{eq:uk displacmenet}
	u_k  =  j(d_k)
\end{equation}
% 
\begin{IEEEeqnarray*}{CCCCCCCCC} \label{eq:uk displacement2}
	u_k & = & 
		\begin{bmatrix}
			u_{[r],k}\\
			u_{[ \phi],k}\\
		\end{bmatrix}
\end{IEEEeqnarray*} 
 
Given $x_{k-1}$ and $u_{k-1}$, the next location of the robot, $x_k$ can be computed.  
\begin{IEEEeqnarray}{CCCCCCCCC} \label{eq:xk function}
	x_k & = & f(x_{k-1}, u_{k-1}) & = & 
	\begin{bmatrix}
		f_x(x_{k-1}, u_{k-1})\\
		f_y(x_{k-1}, u_{k-1})\\
		f_\phi(x_{k-1}, u_{k-1})\\
	\end{bmatrix}
\end{IEEEeqnarray} 

In the above derivation of the system model it was assumed that there are no noise sources. In the next section, we model the noise in the system.

\subsection{System Model with noise}\label{s:20-Model--SysModeNoisel}
We assume that the noise in the odometry can be modeled by a random noise vector $q_k$ such that the noise is Gaussian distribution with zero mean,$\hat{q_k}$ and covariance matrix,$U_k$ .

\begin{equation} \label{eq: Noise Model}
q_k \sim   N(\hat{q_k} , U_k) 
\end{equation}

where

\begin{IEEEeqnarray*}{CCCCCCCCC} \label{eq:qk mean}
	\hat{q_k} & = & 
	\begin{bmatrix}
		0\\
		0\\
	\end{bmatrix}
\end{IEEEeqnarray*} 

and 

\begin{equation*} \label{eq: Uk}
	U_k = \mathrm{E}(q_k - \hat{q_k}) (q_k - \hat{q_k})^T  
	\end{equation*}


\begin{IEEEeqnarray*}{CCCCCCCCC} \label{eq:Uk covariance}
	U_k & = &  
	\begin{bmatrix}
		{\sigma}^2_{q[r],k} & {\sigma}_{q[\phi],k} {\sigma}_{q[r],k}\\
		{\sigma}_{q[\phi],k} {\sigma}_{q[r],k} & {\sigma}^2_{q[\phi],k}\\
	\end{bmatrix}
\end{IEEEeqnarray*} 

With the assumption that the noise sources are independent, the off-diagonal elements of the covariance matrix, $U_k$ are equal to zero.The computation of variances $ {\sigma}^2_{q[r],k}  $ and ${\sigma}^2_{q[\phi],k} $ for the model is discussed in section.
 
 The control input, or relative displacement can be expressed now as shown below.
\begin{equation} \label{eq: Uk + noise}
u_k = j(d_k) + q_k  
\end{equation}

\begin{IEEEeqnarray*}{CCCCCCCCC} \label{eq:Uk + noise2}
	u_k & = & 
	\begin{bmatrix}
			u_{[r],k}\\
			u_{[ \phi],k}\\
	\end{bmatrix}
	& + &
	\begin{bmatrix}
			q_{[r],k}\\
			q_{[ \phi],k}\\
	\end{bmatrix}	
\end{IEEEeqnarray*} 

This makes $u_k$ a random vector. Assuming, that $u_{[r],k}$ and $u_{[ \phi],k}$ are deterministic, uncertainty in$ u_k $equals the uncertainty in the noise term $q_k$.

The system noise can similarly be modeled by a random noise vector $w_k$ such that the noise is Gaussian distribution with zero mean , $\hat{w_k}$ and covariance matrix, $Q_k$
\begin{equation} \label{eq: Noise Model w_k}
w_k \sim   N(\hat{w_k} , Q_k) 
\end{equation}

where

\begin{IEEEeqnarray*}{CCCCCCCCC} \label{eq:wk mean}
	\hat{w_k} & = & 
	\begin{bmatrix}
		0\\
		0\\
		0\\
	\end{bmatrix}
\end{IEEEeqnarray*} 

and 

\begin{equation*} \label{eq: Qk}
Q_k = \mathrm{E}(w_k - \hat{w_k}) (w_k - \hat{w_k})^T  
\end{equation*}


\begin{IEEEeqnarray*}{CCCCCCCCC} \label{eq:Qk covariance}
	Q_k & = &  
	\begin{bmatrix}
		{\sigma}^2_{w[x],k} & {\sigma}_{w[y],k} {\sigma}_{w[x],k} &  {\sigma}_{w[\phi],k} {\sigma}_{w[x],k}\\
		{\sigma}_{w[x],k} {\sigma}_{w[y],k} & {\sigma}^2_{w[y],k} & {\sigma}_{w[\phi],k} {\sigma}_{w[y],k}\\
		{\sigma}_{w[x],k} {\sigma}_{w[\phi],k} & {\sigma}_{w[\phi],k} {\sigma}_{w[x],k} & {\sigma}^2_{w[\phi],k}\\
	\end{bmatrix}
\end{IEEEeqnarray*} 

With the assumption that the noise sources are independent, the off-diagonal elements of the covariance matrix, $Q_k$ are equal to zero.The computation of variances $ {\sigma}^2_{w[x],k}  $ , $ {\sigma}^2_{w[y],k}  $ and ${\sigma}^2_{w[\phi],k} $ for the model is discussed in section.

 The system can be expressed now as shown below.
 \begin{equation} \label{eq: xk + wk}
	x_k = f(x_{k-1}, u_{k-1}) + w_k 
 \end{equation}
 
\begin{IEEEeqnarray}{CCCCCCCCC} \label{eq:xk + noise function}
	x_k & = & 
	\begin{bmatrix}
		f_x(x_{k-1}, u_{k-1})\\
		f_y(x_{k-1}, u_{k-1})\\
		f_\phi(x_{k-1}, u_{k-1})\\
	\end{bmatrix}
	& + &
	\begin{bmatrix}
		w_{[x],k-1})\\
		w_{[y],k-1})\\
		w_{[\phi],k-1})\\
	\end{bmatrix}	
\end{IEEEeqnarray} 

$w_k$ consists of noise sources that are not directly related to $u_k$.
Now, $x_k$ is a random vector and with every time step the system noise increases the variance. Thus, the variance of the location grows with every time step. %assuming that $f_x$, $f_y$ and $f_\phi$ are deterministic, 
 
We have proposed the System Model so that we can implement Kalman Filter on it. Now, we need to prepare the Measurement Model for it.


\subsection{Measurement Model with noise}\label{s:20-Model--MeasurementModel}
Starting with the assumption that there is no noise in the measurement, $z_k$, it is simply a vector containing for each state variable a variable that takes on the value of the corresponding state variable.
The measurement vector, $z_k$ is 

\begin{IEEEeqnarray}{CCCCCCCCC} \label{eq:zk}
	z_k & = & 
	\begin{bmatrix}
		z_{[x],k}\\
		z_{[y],k}\\
		z_{[\phi],k}\\ 	
	\end{bmatrix}
\end{IEEEeqnarray} 


In our system, we use the motion capture system in the lab to localize the robot. The motion capture system acts as absolute sensor and provides us directly the position of the robot in x and y coordinates and its orientation. Thus, $z_k$ in this case is simply expressed as \ref{eq:zk 2}

\begin{IEEEeqnarray*}{CCCCCCCCC} \label{eq:zk 2}
	z_k & = & 
	\begin{bmatrix}
		x_{[x],k}\\
		x_{[y],k}\\
		x_{[\phi],k}\\ 	
	\end{bmatrix}
\end{IEEEeqnarray*} 


Now, we can add noise to the measurement model. We assume that the noise in the odometry can be modeled by a random noise vector $v_k$ such that the noise is Gaussian distribution with zero mean,$\hat{v_k}$ and covariance matrix,$R_k$ .

\begin{equation} \label{eq: Noise Model vk}
v_k \sim   N(\hat{v_k} , R_k) 
\end{equation}

where

\begin{IEEEeqnarray*}{CCCCCCCCC} \label{eq:vk mean}
	\hat{v_k} & = & 
	\begin{bmatrix}
		0\\
		0\\
		0\\
	\end{bmatrix}
\end{IEEEeqnarray*} 

and 

\begin{equation*} \label{eq: Rk}
R_k = \mathrm{E}(v_k - \hat{v_k}) (v_k - \hat{v_k})^T  
\end{equation*}

%
\begin{IEEEeqnarray*}{CCCCCCCCC} \label{eq:Rk covariance}
	R_k & = &  
	\begin{bmatrix}
		{\sigma}^2_{v[x],k} & {\sigma}_{v[y],k} {\sigma}_{v[x],k} &  {\sigma}_{v[\phi],k} {\sigma}_{v[x],k}\\
		{\sigma}_{v[x],k} {\sigma}_{v[y],k} & {\sigma}^2_{v[y],k} & {\sigma}_{v[\phi],k} {\sigma}_{v[y],k}\\
		{\sigma}_{v[x],k} {\sigma}_{v[\phi],k} & {\sigma}_{v[\phi],k} {\sigma}_{v[x],k} & {\sigma}^2_{v[\phi],k}\\
	\end{bmatrix}
\end{IEEEeqnarray*} 

With the assumption that the noise sources are independent, the off-diagonal elements of the covariance matrix, $R_k$ are equal to zero.

The measurement model can be expressed now as shown below.
\begin{equation} \label{eq: zk + noise}
z_k = x_k + v_k  
\end{equation}

\begin{IEEEeqnarray*}{CCCCCCCCC} \label{eq:Uk + noise2}
	z_k & = & 
	\begin{bmatrix}
		x_{[x],k}\\
		x_{[y],k}\\
		x_{[\phi],k}\\ 
	\end{bmatrix}
	& + &
	\begin{bmatrix}
		v_{[x],k-1})\\
		v_{[y],k-1})\\
		v_{[\phi],k-1})\\
	\end{bmatrix}	
\end{IEEEeqnarray*} 

Since the measurement noise $v_k$ is a Gaussian vector, this makes $z_k$ a random vector.


% % % % CODE REFERENCE
%
%\scalefig{test.jpg}{0.8}{test66}\label{img:img1}
%
%\section{Epipolar geometry}
%\label{s:2}
%\indent\indent
%asd \ref{f:test.jpg} dsa
%
%\begin{enumerate}
%	\item Epipole: Projection of optical centers of the cameras lenses $O$, into the other camera image plane:
%	\begin{enumerate}
%		\item Left epipole: Projection of $O_R$ on the left image plane $e_L$.
%		\item Right epipole: Projection of $O_L$ on the right image plane $e_R$.
%	\end{enumerate}
%	\item Baseline: Line connecting $O_L$ and $O_R$. The baseline intersects each image plane at the epipoles $e_L$ and $e_R$.
%	\item Epipolar plane: Plane containing 3 points in space $X$, $O_L$ and $O_R$.
%	\item Epipolar line: Intersection of the epipolar plane with the image plane.
%\end{enumerate}
%
%The line $O_L\!\--\!X$ is seen by left camera as a point, because it is directly in line with that camera’s centre of projection. This means all the points on this line e.g. $X$, $X_1$, $X_2$, $X_3$ will be projected on $x_L$. However, the right camera sees this line as an actual line in its image plane. The projection of this line is in fact an epipolar line. In the same manner, line $O_R\!\--\!X$ is projected on the epipolar line $x_L\!\--\!e_L$ on the left image plane. 
%
%\begin{equation}\label{e:2}
%	p_r^T F p_l = 0
%\end{equation}
%
%where 
%
%\begin{IEEEeqnarray}{CCCCCCCCC} \label{e:3}
%	p_r & = & 
%	\begin{bmatrix}
%		x_r \\
%		y_r \\
%		1
%	\end{bmatrix} ,  & \phantom{~ ~} & & \phantom{~ ~} &
%	p_l & = & 
%	\begin{bmatrix}
%		x_l \\
%		y_l \\
%		1
%	\end{bmatrix}
%\end{IEEEeqnarray}
%
%If fundamental matrix F is written as shown in \bref{eq:4}
%
%\begin{IEEEeqnarray}{CCCCC} \label{e:4}
%	F & = & 
%	\begin{bmatrix}
%		 f_{11}	& f_{12} & f_{13}\\
%		 f_{21}	& f_{22} & f_{23}\\
%		 f_{31} & f_{32} & f_{33}\\
%	\end{bmatrix},
%\end{IEEEeqnarray}
%
%equation \bref{eq:2} can be rewritten to the from portrayed by equation \bref{eq:5}
%
%asd \ref{f:test.jpg} dsa
%
%\begin{equation} \label{e:5}
%	x_l x_r f_{11} + x_l y_r f_{21} + x_l f_{31} + y_l x_r f_{12} + y_l y_r f_{22} + y_l f_{32} + x_r f_{13} + y_r f_{23} + f_{33} = 0
%\end{equation}
%
%The entries of the fundamental matrix F, can be determined by establishing eight or more correspondences. The equation \bref{eq:5} is rearranged to form a homogeneous system, as shown in \bref{eq:6}.
%
%\begin{equation} \label{e:6}
%A f = 0
%\end{equation}
%
%\begin{IEEEeqnarray}{CCCCCCCCCCCCCCC} \label{e:7}
%	A & = & 
%	\begin{bmatrix}
%		x_{l1} x_{r1}	& x_{l1} y_{r1} &  x_{l1} & y_{l1} x_{r1} & y_{l1} y_{r1} & y_{l1} & x_{r1} & y_{r1} & 1 \\
%		x_{l2} x_{r2}	& x_{l2} y_{r2} &  x_{l2} & y_{l2} x_{r2} & y_{l2} y_{r2} & y_{l2} & x_{r2} & y_{r2} & 1 \\
%		\vdots & \vdots & \vdots & \vdots & \vdots &  \vdots & \vdots & \vdots &  \\
%		x_{ln} x_{rn}	& x_{ln} y_{rn} &  x_{ln} & y_{ln} x_{rn} & y_{ln} y_{rn} & y_{ln} & x_{rn} & y_{rn} & 1 \\
%	\end{bmatrix}
%\end{IEEEeqnarray}
%
%\begin{IEEEeqnarray}{CCCCCCCCC} \label{e:8}
%	f & = & 
%	\begin{bmatrix}
%		f_{11}\\
%		f_{12}\\
%		f_{13}\\
%		f_{21}\\
%		f_{22}\\
%		f_{23}\\
%		f_{31}\\
%		f_{32}\\
%		f_{33}\\
%	\end{bmatrix}
%\end{IEEEeqnarray}
%
%%- We can determine the entries of the matrix F (up to an unknown scale factor) by
%%establishing n ³ 8 correspondences:
%%Ax = 0
%%- It turns out that A is rank deficient (i.e., $rank(A) = 8$); the solution is unique up to a scale factor (i.e., proportional to the last column of $V$ where $A = U D V T$ ).
%
%Matrix $A$ is $n \verb|x| 9$ matrix where $n$ is the number of correspondences. However, the rank of matrix $A$ is $8$, which makes it rank deficient. This gives a unique solution up to a scale factor and the solution is proportional to the last column of $V$, where $A = U D V^T $.
%
%The 8-point algorithm can be summarised in following steps.
%
%\begin{enumerate}
%	\item Take $n$ correspondences from $2$ stereo pair images where $n$ is greater than $8$.
%	\item Construct homogeneous system $Ax = 0$, as described above, where $A$ is an $n \verb|x| 9$ matrix.
%	\item Matrix $F$ can be computed by singular value decomposition (SVD) of matrix $A$. The entries of $F$ are proportional to the components of the last column of $V$.
%	\item Enforce the [singularity] constraint: $rank(F) = 2$ by the following steps:
%	\begin{enumerate}
%		\item Compute the SVD of $F$.
%		\item Set the smallest singular value equal to $0$ and let $D$ be the corrected matrix.
%		\item The corrected estimate of $F$ is given by $F_{e} = U D V^T$.
%	\end{enumerate}
%\end{enumerate}
%
%
%\section{Point normalization}
%\label{s:pointnormalization}
%
%In order to implement the 8-point algorithm and estimate the fundamental matrix $F$, corresponding points in the image are expressed in a vector form of 3 elements, as shown in equations \bref{eq:3} and \bref{eq:4}. The $3^{rd}$ element of the vector is assigned the value of $1$ and this is done to prepare a homogeneous vector. For most of the corresponding points, the first two elements are much larger than the $3^{rd}$ element. However, this results in vector pointing in more or less the same direction for all points. Similarly, the $F$ estimated with this approach is not invariant to point transformations.
%
%In order to make the algorithm numerically stable and the estimation more precise, it is more suitable to use normalized points. This is done by transforming the coordinates of each of the two images independently such that the \textit{average} point is equal to $(1, 1, 1)^T$, which is achieved in 2 steps:
%
%\begin{enumerate}
%	\item The origin of the new coordinate system is centered (has its origin) at the centroid (center of gravity) of the image points. This is accomplished by a translation of the original origin to the new one. 
%	\item After the translation, the coordinates are uniformly scaled, so that their average distance from the origin equals $\sqrt{2}$.
%\end{enumerate}
%
%This principle results in a distinct coordinate transformation for each of the two images. As a result, new homogeneous image coordinates $p_l$, $p_r$, are given by
%
%\begin{equation}\label{e:10}
%\bar{p}_l =  T_l p_l 
%\end{equation}
%\begin{equation}\label{e:11}
%\bar{p}_r =  T_r p_r 
%\end{equation}
%
%where $T_l$, $T_r$ are the normalized transformations, responsible for translation and scaling from the old to the new normalized image coordinates. This normalization is only dependent on the image points, which are used in a single image. Normalized transformation is given by equation \bref{eq:9}
%
%\begin{IEEEeqnarray}{CCCCC}\label{e:9}
%	T & = & 
%	\begin{bmatrix}
%		s & 0 & -sc_x \\
%		0 & s & -sc_y \\
%		0 & 0 & 1 \\
%	\end{bmatrix}
%\end{IEEEeqnarray}
%
%where $c$ is the centroid of all points and $s$ is the scale to make average distance equal to $\sqrt{2}$.
%
%This normalization should be carried out before applying the 8-point algorithm. After the fundamental matrix $\bar{F}$ has been estimated, de-normalization is done to get the results in original coordinates system. $\bar{F}$ can be de-normalized to give $F$ according to \bref{eq:12}.
%
%\begin{equation}\label{e:12}
%F = T_r^T \bar{F} T_l
%\end{equation}
%
%In general, this estimate of the fundamental matrix is a more precise one, than one would have obtained by estimating from not normalized coordinates, even at the price of a few calculations more.
%
%% This normalizing transformation will nullify the effect of the arbitrary selection of origin and scale in the coordinate frame of the image, and will mean that the combined algorithm is invariant to a similarity transformation of the image.
%
%\clearpage
%
%\section{Main script}
%\label{s:mainscript}
%%\lstinputlisting[language=Matlab, firstline=1, firstnumber=1]{L06.m}
%\noindent\noindent
%%\script{L06}{Script calling the implemented 8-point algorithm for two image sets with and without normalization of points.}
%
%\begin{equation}\label{e:epipoles}
%F = U W V^T
%\end{equation}
%
%Additionally an epipolar check has been conducted, which should also result in $0$ in the ideal case, as displayed by equations (\ref{eq:epc1} $\div$ \ref{eq:epci2}). The algorithm of performing the test and printing the results (lines $125 \div 149$) is analogue to that of epipolar constraint check, with the difference that each point set is used with its corresponding epipole. Worth mentioning is the actual value of the tolerance used for this check. For the given image sets, $\epsilon$ can be as low as $\epsilon = 10^{-12}$.
%
%\begin{equation}\label{e:epc1}
%e_L^T F P_1 < \epsilon
%\end{equation}
%\begin{equation}\label{e:epc2}
%e_R^T F^T P_2 < \epsilon
%\end{equation}
%
%\begin{equation}\label{e:epci1}
%e_L^T F x_L = 0
%\end{equation}
%\begin{equation}\label{e:epci2}
%e_R^T F^T x_R = 0
%\end{equation}

%\movie[externalviewer]{abc}{test.mp4}
%\movie[width=4cm, height=3cm, poster, externalviewer]{abc}{test.mp4}

%\includemedia[width=4cm, height=3cm, addresource=test.mp4, flashvars={source=test.mp4}]{}{test.mp4}

%\begin{center}
%	\href{run:mpv test.mp4}{
%		\includegraphics[scale=0.25]
%		{test.jpg}}
%\end{center}

% % % % CODE REFERENCE

\section{Conclusions}
\indent\indent


%\end{Section}
\end{Problem}
%\layout
\end{spacing}
\end{document}


% % % Licence

%----------------------------------------------------------------------%
% The following is copyright and licensing information for
% redistribution of this LaTeX source code; it also includes a liability
% statement. If this source code is not being redistributed to others,
% it may be omitted. It has no effect on the function of the above code.
%----------------------------------------------------------------------%
% Copyright (c) 2007, 2008, 2009, 2010, 2011 by Theodore P. Pavlic
%
% Unless otherwise expressly stated, this work is licensed under the
% Creative Commons Attribution-Noncommercial 3.0 United States License. To
% view a copy of this license, visit
% http://creativecommons.org/licenses/by-nc/3.0/us/ or send a letter to
% Creative Commons, 171 Second Street, Suite 300, San Francisco,
% California, 94105, USA.
%
% THE SOFTWARE IS PROVIDED "AS IS", WITHOUT WARRANTY OF ANY KIND, EXPRESS
% OR IMPLIED, INCLUDING BUT NOT LIMITED TO THE WARRANTIES OF
% MERCHANTABILITY, FITNESS FOR A PARTICULAR PURPOSE AND NONINFRINGEMENT.
% IN NO EVENT SHALL THE AUTHORS OR COPYRIGHT HOLDERS BE LIABLE FOR ANY
% CLAIM, DAMAGES OR OTHER LIABILITY, WHETHER IN AN ACTION OF CONTRACT,
% TORT OR OTHERWISE, ARISING FROM, OUT OF OR IN CONNECTION WITH THE
% SOFTWARE OR THE USE OR OTHER DEALINGS IN THE SOFTWARE.
%----------------------------------------------------------------------%